% 本文件是示例论文的一部分
%第三章 需求分析
% 论文的主文件位于上级目录的 `main.tex`
\chapter{雨水资源GIS专题图需求分析}

数学公式是\LaTeX{}排版中举世闻名的强项,关于数学公式的排版。在此,本文无意
展开讨论\LaTeX{}中的数学公式排版。只是重点说明一下由 \nwafuthesis{} 提供的
特有的宏。

\section{专题图总体概述}
\subsection{系统属性}
\subsection{开发背景}
\subsection{功能概述}
\subsection{用户特点}

对于数学公式的排版在\enquote{lshort-zh-cn.pdf}的第四章给出了基本的使用方法,
请大家阅读学习。其内容对大多数人来说已经足够用了,但是如果不能解决问题的话
建议大家求助于搜索引擎或者有经验的人,这也不失为一个好办法。

常见的几个学习\LaTeX{}数学公式排版的资源链接如下:

\begin{itemize}
  \item 数学排版常见问题集:
        \url{https://www.latexstudio.net/index/details/index/mid/635}
  \item \pkg{amsmath}手册中译:
        \url{https://www.latexstudio.net/index/details/index/mid/706}
  \item \LaTeX{}公式备忘单:
        \url{https://www.latexstudio.net/index/details/index/mid/1052.html}
\end{itemize}

\section{具体需求}
\subsection{功能需求}
\subsection{性能需求}
\subsection{界面需求}
\subsection{接口需求}

按我校学位论文排版要求,公式排版需要行间居中排版,公式编号按照一级标题(章)
连续编号(按章)并加小括号,不加导引线。类似这些细节\nwafuthesis{}模板都已进
行了设置。在撰写论文中只要将公式置于\env{equation}环境,并用\cs{label}命令
添加标签后用\cs{ref}或\cs{eqref}命令引用该公式即可。对于多行公式可以在
\env{equation}环境中使用\env{aligned}环境实现排版。

需要注意的是,公式解释下面的\enquote{式中}两字需要左起顶格编排,后接符号及
其解释;解释顺序为先左后右,先上后下;解释与解释之间用中文分号“;”分隔。
此时可以用\cs{noindent}命令临时取消首行缩进,在解释完公式符号后,再次正常
用空行进行分段便可自动恢复段落首行缩进。

例如:勾股定理可以表示为\ref{eq:gougu}

\begin{equation}
  a^2+b^2=c^2\label{eq:gougu}
\end{equation}

\noindent
式中,$a$是一条直角边边长;$b$是另一条直角边边长;$c$是斜边边长。

在公式解释结束后,段落缩进应复位至首行缩进2个汉字的模式。




%%% Local Variables: 
%%% mode: latex
%%% TeX-master: "../main.tex"
%%% End:
