% 本文件是示例论文的一部分
%第六章 总结与展望
% 论文的主文件位于上级目录的 `main.tex`

\chapter{总结与展望}



\section{总结}
时光荏苒,大学马上要一去不复返了,回首过去的一年,内心不禁感慨万千,从选题到完成毕业论文接近一年的时间,通过这一年的毕业论文的设计完成,让我跟着导师学习了很多知识,让自己也得到了提高,这篇论文的题目是基于openlayers与geoserver的雨水资源GIS专题图设计与实现,刚开始对于这个题目不是很了解,因为大学中没有怎么接触相关的内容与知识点,但是对其还是挺感兴趣的,便选择了这个题目。

虽然自己只是负责系统的一小部分内容,但是通过对这个题目的学习深入,让自己深刻的学习了一个系统从有想法到落地所有经过的过程,想出一个课题,首先我们需要清楚该课题的研究背景,通过研究背景才能更好的做需求分析,更好的明白我们做这一项目的意义,然后我们需要查阅资料,了解国内外的研究现状,不能关起门来做项目,这也是导师告诉我们的,通过跟着导师学习,深刻的明白了完成一个系统第一步应该做的是什么。导师的一个团队经常聚在一起讨论需求,需求是完成系统的前提,通过和导师还有身边一起做的同学交流,让自己对于需求分析有了系统性的学习与了解,自己也明白了为什么做需求分析,如何做需求分析等问题,我们通过需求分析,整理出需要做的事情,需要完成的功能等,然后通过我们整理出的这些资料进行技术的选取和使用,看那些技术能更好的做出我们想要的结果,最后对于我的课题而言,我们选择了geoserver与openlayers两个发布服务与地图渲染的技术来进行开发。

选择完技术之后,便是开始做GIS专题图的总体设计,从设计思路,到功能设计以及数据库设计,让我体会到一个系统的雏形已经形成,这一步感觉自己以前的知识也用上了,之前学过数据库,通过数据库的基本知识,设计出这个系统的数据库感觉十分有成就感,通过总体设计这一环节,自己也学习到了一个系统的初步架构是如何形成的,让自己的全局观,从整体上开考虑问题的能力得到了切实的提高,我也觉得这种能力正式我之前所欠缺的。

当总体设计结束之后,便是GIS专题图的详细设计,我们使用QGIS画底图,设计图层,然后发布到geoserver上,然后通过openlayers获取发布到geoserver上的地图,然后对地图进行渲染得到需求所需要的结果,通过不同类型专题图的分类与实现,而且还得到了老师与同学的帮助。最终将所有的专题图实现了出来。

毕业论文完成并不是终点,而是一个新的起点,自己一定会继续学习,再接再厉,更上一层楼。

\section{展望}

基于openlayers与geoserver的雨水资源GIS专题图的设计与实现设计多方面的理论、方法和技术。仅仅通过这一题目的研究与学习还无法真正的对知识有很全面的了解,因此基于openlayers与geoserver的雨水资源专题图的设计与实现还有很多问题需要解决,需要在实际应用中不断积累与完善。而且随着时代的发展,气候的变化等等,势必会对各个业务的对接提出新的需求,这也需要我们去不断地探索新的需求,发展应用新的技术等等,使得系统更加完善,集成度更加深入。在以下几个方面中还需要做进一步的研究与开发。

1.论文对于雨水资源最主要最基本的需求做了设计与实现,对于更加深入的信息如光辐射紫外线的强弱等等还需要加强,可以咨询气候的相关专家了解光辐射如何更好地进行设计与实现,增加系统功能的完整性。

2.论文只考虑了需求分析的几种情况,对于后续可能需要增加的功能的可扩展性不高,这也是对于以后而言需要改进的内容。

3.在设计过程中,有些地理底图较大,占内存较高,对于一些性能不太好的设备而言,加载可能需要一些时间,对于设备的运行内存的要求也比较高,因此如何将地理底图进行金字塔分割,根据比例尺进行分层实现,这也是接下来需要考虑的内容。

4.该系统由于地理气候等各种数据的限制问题,未能按照某一市级区域进行设计,只能选取了中国地图数据较为获取的地图作为本次题目的设计底图,接下来需要改进的就是获取到相关区域的具体地理气候等数据,进行二次开发。




%%% Local Variables: 
%%% mode: latex
%%% TeX-master: "../main.tex"
%%% End:
