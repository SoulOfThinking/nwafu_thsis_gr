% 英文摘要
"Rainwater resources monitoring and management system" is a combination of meteorological stations, moisture stations, GIS geographic systems and other monitoring and sensing equipment to implement weather, moisture, surface water resources (including reservoirs, reservoirs, rivers, rainwater) and other data monitoring and transmission system, the system combined with computer software, for regional rainwater resources and water structure for collaborative management, to assist local governments and enterprises to make full use of and develop local rainwater resources, in order to solve the current situation of extremely unbalanced distribution of water resources in the northwest region.

This project is a module of the rainwater resources monitoring and management system, which mainly displays the types of water resources and the distribution of water resources in the specified areas in the form of GIS thematic maps. The system selects the technical means combined with GeoServer and OpenLayers to implement the GIS thematic map, and clarifies the functions that need to be completed and the design of the database through requirements analysis. In terms of implementation, QGIS is used to implement gis thematic map basemap, save memory and improve efficiency by publishing tile maps, classify and publish various data through GeoServer, use OpenLayers to render the basemap obtained from GeoServer, and use 3D maps for GIS thematic maps for specific needs. The final GIS thematic map is intuitive and concise, which can directly and conveniently observe various information, and can rationally allocate and utilize water resources based on this thematic map.