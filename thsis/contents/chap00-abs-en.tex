% 英文摘要
"Rainwater resource monitoring and management system" is a combination of monitoring and sensing equipment such as weather stations, moisture stations, GIS geographic systems, etc., to monitor and monitor weather, moisture, surface water resources (including reservoirs, reservoirs, rivers, rainwater) and other data. Transmission, combined with computer software, conducts collaborative management of regional rainwater resources and water use structure, assisting local governments and enterprises to fully utilize and develop local rainwater resources.

The system is a subsystem of the rainwater resource monitoring and management system, which is used to display the types of water resources in a designated area and the distribution of water resources to users in the form of GIS thematic maps. The design and implementation of the system are mainly based on the understanding of the research background in the early stage, clarifying the meaning of the topic, and then collecting information in this field to grasp the current research status at home and abroad, and then through the analysis of the specific needs of this project, the geoserver and openlayers are selected. Combining the technical means to realize the GIS thematic map, and through the demand analysis, it clarifies the functions that the system needs to complete, the content that the GIS thematic map needs to display, and then completes the design content of the database. In terms of implementation, QGIS is used to realize the basemap of GIS thematic maps, save memory and improve efficiency by publishing tile maps, classify and publish various data through geoserver, and use openlayers to render the basemap obtained from geoserver. The GIS thematic map for specific needs is realized by 3D map, and the display content required by the demand analysis is obtained. The final GIS thematic map is intuitive and concise, which enables users to observe various information directly and conveniently.