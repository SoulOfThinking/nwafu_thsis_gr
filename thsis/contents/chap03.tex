% 本文件是示例论文的一部分
%第三章 需求分析
% 论文的主文件位于上级目录的 `main.tex`
\chapter{雨水资源GIS专题图需求分析}



\section{专题图总体概述}
GIS专题地图是以普通地图为地理基础,着重表示制图区域内某一种或几种自然要素或社会经济现象的地图。

专题地图的内容主要由两部分构成:专题内容和地理基础,前者为地图上突出表示的自然要素或社会经济现象及其有关特征;后者为用以表明专题图要素空间位置与地理背景的普通地图内容。

这类地图的显示特点是,作为该图主题的专题图内容予以详尽表示,其地理基础内容则视主题而异,有选择地表示某些相关要素。

因此我们通过对具体问题进行具体分析,分析得到本项目的具体需求,以便设计出专题更加鲜明,符合各个要求需求的GIS专题图。

\subsection{系统属性}
该系统是雨水资源监控管理系统的一个子系统,用于将指定区域的水资源种类情况以及水资源分布情况以GIS专题图形式展示给用户,能够让用户更清晰、直观的看到其所关心的部分,方便用户更好的协调各个区域的水资源调用实现水资源的高效利用。
\subsection{开发背景}
开发目的是为了方便用户查看水资源的收集、分布等情况,通过GIS专题图形式为用户提供可视化方便理解的信息,为水资源的调度提供决策支持,应用的目标是使该系统的操作人员以及该系统覆盖地区下的所有居民、政府人员等可以直观地了解各个地区的水资源储量等信息。
\subsection{功能概述}
主要功能为以GIS专题图的形式展示土壤类型、肥力、墒情、酸碱度等信息;展示空气温湿度、风速风向、光照等信息;展示各个地区所属的地下水、水库、河流等水量信息;一起后信息和植被覆盖信息等为参考,展示各个地区的蒸散发信息。
\subsection{用户特点}
主要使用人员为相关工作人员、技术人员、政府人员、企业人员以及定边县的居民等。
其中相关工作人员、技术人员、政府人员、企业人员具备相关技能知识培训且有使用过类似的系统的操作经验,对于本系统可以很快上手使用,而对于普通居民用户需要进行必要的操作培训才能灵活使用该系统。


%对于数学公式的排版在\enquote{lshort-zh-cn.pdf}的第四章给出了基本的使用方法,
%请大家阅读学习。其内容对大多数人来说已经足够用了,但是如果不能解决问题的话
%建议大家求助于搜索引擎或者有经验的人,这也不失为一个好办法。
%
%常见的几个学习\LaTeX{}数学公式排版的资源链接如下:
%
%\begin{itemize}
%  \item 数学排版常见问题集:
%        \url{https://www.latexstudio.net/index/details/index/mid/635}
%  \item \pkg{amsmath}手册中译:
%        \url{https://www.latexstudio.net/index/details/index/mid/706}
%  \item \LaTeX{}公式备忘单:
%        \url{https://www.latexstudio.net/index/details/index/mid/1052.html}
%\end{itemize}

\section{具体需求}
\subsection{功能需求}
\begin{enumerate}
	\item   土壤信息:
	以GIS专题图的形式展示土壤类型、肥力、墒情酸碱度等信息,不同地区的土壤类型用不同颜色表示,肥力墒情和酸碱度信息通过用户点击来显示出某个地区的肥力墒情酸碱度信息。
    \item 气候信息:
    以GIS专题图的形式展示气候信息,主要为空气温湿度,风速风向,光照等信息,在定边县地图为底图,以图标的形式展示风向和光照信息,不同颜色展示温度信息,当点击各种图标时,会显示出风向,光照强度,具体温湿度信息。
    \item 降雨信息:
    以GIS专题图的形式展示各个区域的降雨信息,包括过去、现在、未来等情况。
    \item 蓄水信息:
    使用GIS专题图来展示各个区域的河流水库地下水的蓄水情况,不同地下水储量情况可以使用不同颜色来表示,水库在地图上使用原点标识出来,河流可采用点击各个区域出现具体数据来展示。
    \item 蒸散发信息:
    通过对气候信息和植被覆盖信息的处理来得出各个区域的蒸散发强度信息,采用不同颜色或者蒸发动态图标展示。不同区域有蒸散发具体数据。
\end{enumerate}
\subsection{性能需求}
\subsubsection{可靠性}
要求系统能经受时效和压力测试,系统平台具备不少于100个访问并发的能力。
\subsubsection{效率}
对展示界面要求简洁明了,符合用户的观看习惯,GIS专题图的显示要快速直观。
\subsubsection{安全性}
高安全性,能够防止网站数据被非法篡改。
\subsubsection{可维护性}
方案和系统的架构须紧密跟踪主流技术标准,开放性好,便于系统的升级维护、以及与各种信息系统进行集成。
\subsection{界面需求}
界面需简洁明了,符合用户观看习惯,采用数据大屏的形式展示GIS专题图,并且根据需求可在专题图周围添加数据图表以便更加方便观察各种数据。
\subsection{接口需求}
本项目主要是展示相关地区的地理,雨水资源,气候等信息,所以对于接口来说主要是为主动从后端获取信息,因此通过webservice进行前后端通信,选择json数据进行数据的传输。





%%% Local Variables: 
%%% mode: latex
%%% TeX-master: "../main.tex"
%%% End:
