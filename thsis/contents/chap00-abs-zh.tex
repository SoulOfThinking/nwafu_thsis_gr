% 中文摘要
“雨水资源监控管理系统”是结合气象站、墒情站、GIS地理系统等监测传感设备,对于天气、墒情、地表水资源(包括水库、蓄水池、河流、雨水)等数据进行实施监测与传输,结合计算机软件,对于区域性雨水资源情况和用水结构进行协同管理,辅助当地政府和企业单位充分利用与开发当地雨水资源。

该系统是雨水资源监控管理系统的一个子系统,用于将指定区域的水资源种类情况以及水资源分布情况以GIS专题图形式展示给用户。该系统的设计与实现主要通过前期首先对于研究背景的了解,明晰该题目的意义,然后收集该领域资料,把握当前国内外研究现状,之后通过对于本项目具体需求分析,选取了geoserver与openlayers相结合的技术手段来进行GIS专题图的实现,并且通过需求分析明晰了本系统需要完成的功能,GIS专题图需要展示的内容,进而完成对数据库的设计内容。在实现上,使用QGIS进行GIS专题图底图的实现,通过发布瓦片地图来节省内存、提高效率,将各种数据通过geoserver进行分类发布,采用openlayers对从geoserver获取的底图进行渲染,对于特定需求的GIS专题图采用了3D地图来实现,得到需求分析要求的展示内容。最后得出的GIS专题图直观简洁,可以使用户直接方便观测各种信息。

