% 中文摘要
“雨水资源监控管理系统”是结合气象站、墒情站、GIS地理系统等监测传感设备,对于天气、墒情、地表水资源(包括水库、蓄水池、河流、地面径流雨水)等数据进行实施监测与传输,结合计算机软件,对于区域性雨水资源情况和用水结构进行协同管理,辅助当地政府和企业单位充分利用与开发当地雨水资源。

该系统是雨水资源监控管理系统的一个子系统,用于将定远县各个村庄或指定区域的水资源种类情况以及水资源分布情况以GIS专题图形式展示给用户,能够让用户更清晰、直观的看到其所关心的部分,方便用户更好的协调各个区域的水资源调用实现水资源的高效利用。

开发目的是为了方便用户查看水资源的收集、分布等情况,通过GIS专题图形式为用户提供可视化方便理解的信息,为水资源的调度提供决策支持,应用的目标是使该系统的操作人员以及该系统覆盖地区下的所有居民、政府人员等可以直观地了解各个地区的水资源储量等信息。
