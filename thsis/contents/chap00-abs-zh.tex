% 中文摘要
“雨水资源监控管理系统”是结合气象站、墒情站、GIS地理系统等监测传感设备对天气、墒情、地表水资源(包括水库、蓄水池、河流、雨水)等数据进行实施监测与传输的系统,该系统结合计算机软件,对于区域性雨水资源情况和用水结构进行协同管理,辅助当地政府和企业单位充分利用与开发当地雨水资源,以解决西北地区水资源分布极不平衡的现状。

本课题是雨水资源监控管理系统的一个模块,主要为将指定区域的水资源种类情况以及水资源分布情况以GIS专题图形式展示给用户。该系统选取了GeoServer与OpenLayers相结合的技术手段来进行GIS专题图的实现,并且通过需求分析明晰了本系统需要完成的功能以及完成对数据库的设计。在实现上,使用QGIS进行GIS专题图底图的实现,通过发布瓦片地图来节省内存、提高效率,将各种数据通过GeoServer进行分类发布,采用OpenLayers对从GeoServer获取的底图进行渲染,对于特定需求的GIS专题图采用了3D地图来实现。最后得出的GIS专题图直观简洁,可以使用户直接方便观测各种信息,并可以基于此专题图对水资源进行合理分配和利用。

