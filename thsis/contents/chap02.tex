% 本文件是示例论文的一部分
% 本文件是论文的第二章:技术路线
% 论文的主文件位于上级目录的 `main.tex`

\chapter{实现技术介绍}



\section{GeoServer}
  GeoServer是一个开源项目,是一个用于共享地理空间数据的开源服务器,专为互操作性而设计,他使用开放标准发布来自任何主要空间数据源的数据,其作为社区驱动的项目由不同个人和组织开发、测试和支持,因此我们可以免费使用该项目,而且,该项目具有可以自行修改、自行复制以及再分发的能力。同时,GeoServer还有众多的优点:


\begin{enumerate}
 
   \item 该服务器使用JAVA 语言开发、使用的架构是标准化的J2EE、并且该服务器的开发是基于servlet 和 STRUTS 框架、并且支持当前非常流行的Spring 框架开发;
   \item GeoServer实施WMS和WFS标准,可以创建各种输出格式的地图、并且支持WFS—T规范;
   \item GeoServer支持各种数据库,如MapInfo、Shapefile、PostGIS 、、MySQL、Oracle 、DB2等;
   \item GeoServer支持上百种坐标系投影;
   \item GeoServer能够将自己制作的地图或者网络地图以KML、png、SVG、GML、gif、jpeg 等格式输出;
   \item 因为该软件的开发使用J2EE架构以及采用了servlet等框架,所以其可以在任何基于J2EE/Servlet的容器上运行;
   \item GeoServer还嵌入MapBuilder,可以将发布的地图拆分为切片并且支持AJAX;
   \item GeoServer可以在线编辑空间数据、并且生成专题地图;
   \item GeoServer发布地图可以使用xml等文件;
   \item GeoServer还支持GOOGLE MAPS等;
   \item GeoServer还可以在服务器上发布KML 数据,并且可以通过与GoogleEarth进行影像叠加进而作出生动的应用;
\end{enumerate}


\section{Openlayers}
由于浏览器在默认情况下,它对于地图(png/jepg等格式)的显示是静态的,因此如果想要在浏览器中展示交互式的地图会有较大的困难,因为如果实现交互的功能,那么对地图的每一个点击和缩放等操作,地图都要做出正确的反应,所以交互式比较难。

OpenLayers 是一个专为Web GIS 客户端开发提供的JavaScript 类库包,它主要是在开发Web GIS客户端时使用,主要是用于实现标准化格式发布的地图的数据访问。在OpenLayers发布的V2.2版本之后后,OpenLayers已经将他在开发时所用到的Prototype.js组件和自己整合在了一起,并且不断在Prototype.js的基础上完善面向对象的开发,不过对于OpenLayers,能用到Rico的地方不多,目前只是在OpenLayers.Popup.AnchoredBubble类中对于DIV进行圆角化处理。在OpenLayers的V2.4版本之后,其提供了矢量画图的功能,这样对于动态地展现“点、线和面”这样的地理数据十分方便。

对于微软Virtual Earth、高德地图、百度地图、Google Maps等资源OpenLayers也可以实现非常好的支持,OpenLayers可以通过WMS等服务调用其它服务器上的空间数据,通过WFS服务调用空间服务。在操作方面,OpenLayers可以在浏览器中实现地图浏览的基本效果,例如缩小、放大、平移等操作,还可以进行选取线、要素选择、图层叠加、选取面等操作,甚至可以对已有的OpenLayers 操作和数据支持类型进行扩充,为其赋予更多的功能。例如,它可以为OpenLayers 添加网络处理服务WPS 的操作接口,从而利用已有的空间分析处理服务来对加载的地理空间数据进行计算。同时,在OpenLayers提供的类库当中,由于整合了类库Prototype.js 和Rico ,因此可以为地图浏览操作客户端增加Ajax 效果。

相对于另一个框架 OpenScales,OpenScales 是 OpenLayers 的 ActionScript 翻译,需要 FlashPlayer 支持才行,所以相比较来说,我们选择了原生的openlayers作为我们的实现技术。




%%% Local Variables: 
%%% mode: latex
%%% TeX-master: "../main.tex"
%%% End:
